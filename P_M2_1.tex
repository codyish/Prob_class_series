% Options for packages loaded elsewhere
\PassOptionsToPackage{unicode}{hyperref}
\PassOptionsToPackage{hyphens}{url}
\PassOptionsToPackage{dvipsnames,svgnames,x11names}{xcolor}
%
\documentclass[
  letterpaper,
  DIV=11,
  numbers=noendperiod]{scrartcl}

\usepackage{amsmath,amssymb}
\usepackage{iftex}
\ifPDFTeX
  \usepackage[T1]{fontenc}
  \usepackage[utf8]{inputenc}
  \usepackage{textcomp} % provide euro and other symbols
\else % if luatex or xetex
  \usepackage{unicode-math}
  \defaultfontfeatures{Scale=MatchLowercase}
  \defaultfontfeatures[\rmfamily]{Ligatures=TeX,Scale=1}
\fi
\usepackage{lmodern}
\ifPDFTeX\else  
    % xetex/luatex font selection
\fi
% Use upquote if available, for straight quotes in verbatim environments
\IfFileExists{upquote.sty}{\usepackage{upquote}}{}
\IfFileExists{microtype.sty}{% use microtype if available
  \usepackage[]{microtype}
  \UseMicrotypeSet[protrusion]{basicmath} % disable protrusion for tt fonts
}{}
\makeatletter
\@ifundefined{KOMAClassName}{% if non-KOMA class
  \IfFileExists{parskip.sty}{%
    \usepackage{parskip}
  }{% else
    \setlength{\parindent}{0pt}
    \setlength{\parskip}{6pt plus 2pt minus 1pt}}
}{% if KOMA class
  \KOMAoptions{parskip=half}}
\makeatother
\usepackage{xcolor}
\setlength{\emergencystretch}{3em} % prevent overfull lines
\setcounter{secnumdepth}{-\maxdimen} % remove section numbering
% Make \paragraph and \subparagraph free-standing
\ifx\paragraph\undefined\else
  \let\oldparagraph\paragraph
  \renewcommand{\paragraph}[1]{\oldparagraph{#1}\mbox{}}
\fi
\ifx\subparagraph\undefined\else
  \let\oldsubparagraph\subparagraph
  \renewcommand{\subparagraph}[1]{\oldsubparagraph{#1}\mbox{}}
\fi

\usepackage{color}
\usepackage{fancyvrb}
\newcommand{\VerbBar}{|}
\newcommand{\VERB}{\Verb[commandchars=\\\{\}]}
\DefineVerbatimEnvironment{Highlighting}{Verbatim}{commandchars=\\\{\}}
% Add ',fontsize=\small' for more characters per line
\usepackage{framed}
\definecolor{shadecolor}{RGB}{241,243,245}
\newenvironment{Shaded}{\begin{snugshade}}{\end{snugshade}}
\newcommand{\AlertTok}[1]{\textcolor[rgb]{0.68,0.00,0.00}{#1}}
\newcommand{\AnnotationTok}[1]{\textcolor[rgb]{0.37,0.37,0.37}{#1}}
\newcommand{\AttributeTok}[1]{\textcolor[rgb]{0.40,0.45,0.13}{#1}}
\newcommand{\BaseNTok}[1]{\textcolor[rgb]{0.68,0.00,0.00}{#1}}
\newcommand{\BuiltInTok}[1]{\textcolor[rgb]{0.00,0.23,0.31}{#1}}
\newcommand{\CharTok}[1]{\textcolor[rgb]{0.13,0.47,0.30}{#1}}
\newcommand{\CommentTok}[1]{\textcolor[rgb]{0.37,0.37,0.37}{#1}}
\newcommand{\CommentVarTok}[1]{\textcolor[rgb]{0.37,0.37,0.37}{\textit{#1}}}
\newcommand{\ConstantTok}[1]{\textcolor[rgb]{0.56,0.35,0.01}{#1}}
\newcommand{\ControlFlowTok}[1]{\textcolor[rgb]{0.00,0.23,0.31}{#1}}
\newcommand{\DataTypeTok}[1]{\textcolor[rgb]{0.68,0.00,0.00}{#1}}
\newcommand{\DecValTok}[1]{\textcolor[rgb]{0.68,0.00,0.00}{#1}}
\newcommand{\DocumentationTok}[1]{\textcolor[rgb]{0.37,0.37,0.37}{\textit{#1}}}
\newcommand{\ErrorTok}[1]{\textcolor[rgb]{0.68,0.00,0.00}{#1}}
\newcommand{\ExtensionTok}[1]{\textcolor[rgb]{0.00,0.23,0.31}{#1}}
\newcommand{\FloatTok}[1]{\textcolor[rgb]{0.68,0.00,0.00}{#1}}
\newcommand{\FunctionTok}[1]{\textcolor[rgb]{0.28,0.35,0.67}{#1}}
\newcommand{\ImportTok}[1]{\textcolor[rgb]{0.00,0.46,0.62}{#1}}
\newcommand{\InformationTok}[1]{\textcolor[rgb]{0.37,0.37,0.37}{#1}}
\newcommand{\KeywordTok}[1]{\textcolor[rgb]{0.00,0.23,0.31}{#1}}
\newcommand{\NormalTok}[1]{\textcolor[rgb]{0.00,0.23,0.31}{#1}}
\newcommand{\OperatorTok}[1]{\textcolor[rgb]{0.37,0.37,0.37}{#1}}
\newcommand{\OtherTok}[1]{\textcolor[rgb]{0.00,0.23,0.31}{#1}}
\newcommand{\PreprocessorTok}[1]{\textcolor[rgb]{0.68,0.00,0.00}{#1}}
\newcommand{\RegionMarkerTok}[1]{\textcolor[rgb]{0.00,0.23,0.31}{#1}}
\newcommand{\SpecialCharTok}[1]{\textcolor[rgb]{0.37,0.37,0.37}{#1}}
\newcommand{\SpecialStringTok}[1]{\textcolor[rgb]{0.13,0.47,0.30}{#1}}
\newcommand{\StringTok}[1]{\textcolor[rgb]{0.13,0.47,0.30}{#1}}
\newcommand{\VariableTok}[1]{\textcolor[rgb]{0.07,0.07,0.07}{#1}}
\newcommand{\VerbatimStringTok}[1]{\textcolor[rgb]{0.13,0.47,0.30}{#1}}
\newcommand{\WarningTok}[1]{\textcolor[rgb]{0.37,0.37,0.37}{\textit{#1}}}

\providecommand{\tightlist}{%
  \setlength{\itemsep}{0pt}\setlength{\parskip}{0pt}}\usepackage{longtable,booktabs,array}
\usepackage{calc} % for calculating minipage widths
% Correct order of tables after \paragraph or \subparagraph
\usepackage{etoolbox}
\makeatletter
\patchcmd\longtable{\par}{\if@noskipsec\mbox{}\fi\par}{}{}
\makeatother
% Allow footnotes in longtable head/foot
\IfFileExists{footnotehyper.sty}{\usepackage{footnotehyper}}{\usepackage{footnote}}
\makesavenoteenv{longtable}
\usepackage{graphicx}
\makeatletter
\def\maxwidth{\ifdim\Gin@nat@width>\linewidth\linewidth\else\Gin@nat@width\fi}
\def\maxheight{\ifdim\Gin@nat@height>\textheight\textheight\else\Gin@nat@height\fi}
\makeatother
% Scale images if necessary, so that they will not overflow the page
% margins by default, and it is still possible to overwrite the defaults
% using explicit options in \includegraphics[width, height, ...]{}
\setkeys{Gin}{width=\maxwidth,height=\maxheight,keepaspectratio}
% Set default figure placement to htbp
\makeatletter
\def\fps@figure{htbp}
\makeatother

\KOMAoption{captions}{tableheading}
\makeatletter
\@ifpackageloaded{caption}{}{\usepackage{caption}}
\AtBeginDocument{%
\ifdefined\contentsname
  \renewcommand*\contentsname{Table of contents}
\else
  \newcommand\contentsname{Table of contents}
\fi
\ifdefined\listfigurename
  \renewcommand*\listfigurename{List of Figures}
\else
  \newcommand\listfigurename{List of Figures}
\fi
\ifdefined\listtablename
  \renewcommand*\listtablename{List of Tables}
\else
  \newcommand\listtablename{List of Tables}
\fi
\ifdefined\figurename
  \renewcommand*\figurename{Figure}
\else
  \newcommand\figurename{Figure}
\fi
\ifdefined\tablename
  \renewcommand*\tablename{Table}
\else
  \newcommand\tablename{Table}
\fi
}
\@ifpackageloaded{float}{}{\usepackage{float}}
\floatstyle{ruled}
\@ifundefined{c@chapter}{\newfloat{codelisting}{h}{lop}}{\newfloat{codelisting}{h}{lop}[chapter]}
\floatname{codelisting}{Listing}
\newcommand*\listoflistings{\listof{codelisting}{List of Listings}}
\makeatother
\makeatletter
\@ifpackageloaded{caption}{}{\usepackage{caption}}
\@ifpackageloaded{subcaption}{}{\usepackage{subcaption}}
\makeatother
\makeatletter
\makeatother
\ifLuaTeX
  \usepackage{selnolig}  % disable illegal ligatures
\fi
\IfFileExists{bookmark.sty}{\usepackage{bookmark}}{\usepackage{hyperref}}
\IfFileExists{xurl.sty}{\usepackage{xurl}}{} % add URL line breaks if available
\urlstyle{same} % disable monospaced font for URLs
\hypersetup{
  pdftitle={Problem 1},
  colorlinks=true,
  linkcolor={blue},
  filecolor={Maroon},
  citecolor={Blue},
  urlcolor={Blue},
  pdfcreator={LaTeX via pandoc}}

\title{Problem 1}
\author{}
\date{}

\begin{document}
\maketitle
This assignment will be reviewed by peers based upon a given rubric.
Make sure to keep your answers clear and concise while demonstrating an
understanding of the material. Be sure to give all requested information
in markdown cells. It is recommended to utilize Latex.

What does it mean for one event \(C\) to cause another event \(E\) ---
for example, smoking (\(C\)) to cause cancer (\(E\))? There is a long
history in philosophy, statistics, and the sciences of trying to clearly
analyze the concept of a cause. One tradition says that causes raise the
probability of their effects; we may write this symbolically is \[
%\begin{equation} 
P(E | C)>P(E). 
%\end{equation} 
\qquad \qquad (1)
\]

\paragraph{Part a)}\label{part-a}

Does equation (1) imply that \(P(C | E) > P(C)\)? If so, prove it. If
not, give a counter example.

\begin{longtable}[]{@{}
  >{\raggedright\arraybackslash}p{(\columnwidth - 0\tabcolsep) * \real{0.0833}}@{}}
\toprule\noalign{}
\begin{minipage}[b]{\linewidth}\raggedright
Answer
\end{minipage} \\
\midrule\noalign{}
\endhead
\bottomrule\noalign{}
\endlastfoot
Answer \\
If the probbility of an event given a causative condition is higher than
the total probability of an event (Equation 1) then the probability of
the event occuring given that condition is necessarily higher than the
probability of the event given the condition does not occur (Equation
2). This can be seen from the law of total probability. \\
\(P(E)=P(E|C)\cdot P(C)+P(E|C^C)\cdot P(C^C)\) \\
If \(P(E|C) > P(E)\) then the \(P(E|C)\) sets the highest possible value
for \(P(E)\) across all values of \(P(C)\), and \(P(C^C)\) sets the
lowest possible value for \(P(E)\). Since \(P(E)\) is the average of
\(P(E|C)\) and \(P(E|C^C)\) (weighted by the given value of \(P(C)\))
then \(P(E|C^C)\) must be lower than \(P(E|C)\) to ``pull'' \(P(E)\)
down from the upper bound of \(P(E|C)\). \\
\#\#\#\# Part c) \\
Let \(C\) be the drop in the level of mercury in a barometer and let
\(E\) be a storm. Briefly describe why this leads to a problem with
using equation (1) (or equation (2)) as a theory of causation. \\
\end{longtable}

Answer

Neither equation 1 nor 2 account for the possibility of a third
condition that influences E and C while C and E don't influence each
other. If C were strictly causal for E then it would be possible to
induce a storm by artificially lowering the mercury in the barometer.

\paragraph{Part d)}\label{part-d}

Let \(A\), \(C\), and \(E\) be events. If
\(P(E | A \cap C) = P(E |C )\), then \(C\) is said to screen \(A\) off
from \(E\). Suppose that \(P (E \cap C) > 0.\) Show that screening off
is equivalent to saying that \(P(A \cap E | C)=P(A|C)P(E | C).\) What
does this latter equation say in terms of independence?

\begin{longtable}[]{@{}
  >{\raggedright\arraybackslash}p{(\columnwidth - 0\tabcolsep) * \real{0.0694}}@{}}
\toprule\noalign{}
\begin{minipage}[b]{\linewidth}\raggedright
Answer
\end{minipage} \\
\midrule\noalign{}
\endhead
\bottomrule\noalign{}
\endlastfoot
Answer \\
Yes, this shows how the drop in the barometer can provide information
about the probability of a storm occurring without out being the cause
of the storm by introducing the third condition (the drop in atmospheric
pressure) that affects the barometer and the weather similarly. \\
\# Problem 2 \\
Suppose you have two bags of marbles that are in a box. Bag 1 contains 7
white marbles, 6 black marbles, and 3 gold marbles. Bag 2 contains 4
white marbles, 5 black marbles, and 15 gold marbles. The probability of
grabbing the Bag 1 from the box is twice the probability of grabbing the
Bag 2. \\
If you close your eyes, grab a bag from the box, and then grab a marble
from that bag, what is the probability that it is gold? \\
\textbf{Part a)} \\
Solve this problem by hand. This should give us a theoretical value for
pulling a gold marble. \\
\end{longtable}

Answer

\begin{itemize}
\tightlist
\item
  Bag 1 - 7 white, 6 black, 3 gold = 16 total
\item
  Bag 2 - 4 white, 5 black, 15 gold = 24 total
\item
  P(Bag1) = 2/3
\item
  P(Bag2) = 1/3
\end{itemize}

\[P(\text{Gold}) = \left(P(\text{Bag1})P(\text{Gold1})\right) + \left(P(\text{Bag2})P(\text{Gold2})\right)\]

\[P(\text{Gold}) = \left(\frac{2}{3}\right)\left(\frac{3}{16}\right) + \left(\frac{1}{3}\right)\left(\frac{15}{24}\right) = \frac{1}{3}\]

\textbf{Part b)}

Create a simulation to estimate the probability of pulling a gold
marble. Assume you put the marble back in the bag each time you pull one
out. Make sure to run the simulation enough times to be confident in
your final result.

Note: To generate \(n\) random values between {[}0,1{]}, use the
\texttt{runif(n)} function. This function generates \(n\) random
variables from the Uniform(0,1) distribution, which we will learn more
about later in this course!

\begin{Shaded}
\begin{Highlighting}[]
\CommentTok{\# set number of simulations}
\NormalTok{sims }\OtherTok{=} \DecValTok{1000000}
\CommentTok{\# create vector to hold results of gold marble pulls}
\NormalTok{gold }\OtherTok{=} \FunctionTok{rep}\NormalTok{(}\DecValTok{0}\NormalTok{, sims)}

\ControlFlowTok{for}\NormalTok{ (i }\ControlFlowTok{in} \DecValTok{1}\SpecialCharTok{:}\NormalTok{sims) \{}
    \CommentTok{\# simulate choose one of the bags}
\NormalTok{    bag\_choice }\OtherTok{=} \FunctionTok{runif}\NormalTok{(}\DecValTok{1}\NormalTok{)}

    \CommentTok{\# define and simulate gold pull probability for bag 2}
    \ControlFlowTok{if}\NormalTok{ (bag\_choice }\SpecialCharTok{\textless{}=} \DecValTok{1}\SpecialCharTok{/}\DecValTok{3}\NormalTok{) \{}
\NormalTok{        pull\_bag\_1 }\OtherTok{=} \FunctionTok{runif}\NormalTok{(}\DecValTok{1}\NormalTok{)}
        \ControlFlowTok{if}\NormalTok{ (pull\_bag\_1 }\SpecialCharTok{\textless{}=} \DecValTok{15}\SpecialCharTok{/}\DecValTok{24}\NormalTok{) \{}
\NormalTok{            gold[i] }\OtherTok{=} \DecValTok{1}
\NormalTok{        \}}
    \CommentTok{\# define and simulate gold pull probability for bag 1}
\NormalTok{    \} }\ControlFlowTok{else}\NormalTok{ \{}
\NormalTok{        pull\_bag\_2 }\OtherTok{=} \FunctionTok{runif}\NormalTok{(}\DecValTok{1}\NormalTok{)}
        \ControlFlowTok{if}\NormalTok{ (pull\_bag\_2 }\SpecialCharTok{\textless{}=} \DecValTok{3}\SpecialCharTok{/}\DecValTok{16}\NormalTok{) \{}
\NormalTok{            gold[i] }\OtherTok{=} \DecValTok{1}
\NormalTok{        \}}
\NormalTok{    \}}
\NormalTok{\}}

\NormalTok{result }\OtherTok{\textless{}{-}} \FunctionTok{mean}\NormalTok{(gold)}
\end{Highlighting}
\end{Shaded}

\begin{Shaded}
\begin{Highlighting}[]
\NormalTok{expected\_value }\OtherTok{\textless{}{-}} \DecValTok{1} \SpecialCharTok{/} \DecValTok{3}
\NormalTok{result\_error }\OtherTok{\textless{}{-}} \FunctionTok{abs}\NormalTok{(}\DecValTok{1} \SpecialCharTok{{-}}\NormalTok{ (result}\SpecialCharTok{/}\NormalTok{expected\_value)) }\SpecialCharTok{*} \DecValTok{100}

\FunctionTok{print}\NormalTok{(}\FunctionTok{sprintf}\NormalTok{(}\StringTok{"The average of \%d simulations is \%f which is within \%.3f\%\% of the expected result of 1/3 "}\NormalTok{, sims, result, result\_error))}
\end{Highlighting}
\end{Shaded}

\begin{verbatim}
[1] "The average of 1000000 simulations is 0.333740 which is within 0.122% of the expected result of 1/3 "
\end{verbatim}

\section{Problem 3}\label{problem-3}

Suppose you roll a fair die two times. Let \(A\) be the event ``the sum
of the throws equals 5'' and \(B\) be the event ``at least one of the
throws is a \(4\)''.

\textbf{Part a)}

By hand, solve for the probability that the sum of the throws equals 5,
given that at least one of the throws is a 4. That is, solve \(P(A|B)\).

\begin{center}\rule{0.5\linewidth}{0.5pt}\end{center}

Answer

\(P(A|B) = \frac{P(A\cap B)}{P(B)}\)

\(P(A) = \frac{1}{9}\)

\(P(B) = 1-(\frac{5}{6})^2 = 11/36\)

\(P(A \cap B) = P(A)\cdot P(B|A)\)

\(P(A \cap B) = \frac{1}{9} \cdot \frac{1}{2} = \frac{1}{18}\)

\$P(A\textbar B) = \frac{\frac {1}{18}} \{\frac{11}{36}\} = \frac{2}{11}
\approx 0.1818 \$

\textbf{Part b)}

Write a simple simulation to confirm our result. Make sure you run your
simulation enough times to be confident in your result.

Hint: Think about the definition of conditional probability.

\begin{Shaded}
\begin{Highlighting}[]
\CommentTok{\# set number of simulations}
\NormalTok{sims }\OtherTok{\textless{}{-}} \DecValTok{1000000}

\CommentTok{\# create vector to count the number of rolls that contain a 4}
\NormalTok{contains\_4 }\OtherTok{\textless{}{-}} \FunctionTok{rep}\NormalTok{(}\DecValTok{0}\NormalTok{, sims)}

\CommentTok{\# create a vector to count the number of rolls that contain a 4 and sum to 5}
\NormalTok{sum\_5\_given\_4 }\OtherTok{\textless{}{-}} \FunctionTok{rep}\NormalTok{(}\DecValTok{0}\NormalTok{,sims)}

\CommentTok{\# create a dice numbered 1 through 6}
\NormalTok{x }\OtherTok{\textless{}{-}} \FunctionTok{c}\NormalTok{(}\DecValTok{1}\SpecialCharTok{:}\DecValTok{6}\NormalTok{)}

\ControlFlowTok{for}\NormalTok{ (i }\ControlFlowTok{in} \DecValTok{1}\SpecialCharTok{:}\NormalTok{sims) \{}
  \CommentTok{\# roll the dice twice}
\NormalTok{  rolls }\OtherTok{=} \FunctionTok{sample}\NormalTok{(x, }\DecValTok{2}\NormalTok{, }\AttributeTok{replace =} \ConstantTok{TRUE}\NormalTok{)}

    \CommentTok{\# if the roll contains a 4, count it in the contain\_4 vector}
    \ControlFlowTok{if}\NormalTok{ (}\DecValTok{4} \SpecialCharTok{\%in\%}\NormalTok{ rolls)\{}
\NormalTok{      contains\_4[i] }\OtherTok{=} \DecValTok{1}
\NormalTok{    \}}
    
    \CommentTok{\# if there is a 4 and the sum is 5, count it in the sum vector}
    \ControlFlowTok{if}\NormalTok{ ((}\DecValTok{4} \SpecialCharTok{\%in\%}\NormalTok{ rolls) }\SpecialCharTok{\&\&} \FunctionTok{sum}\NormalTok{(rolls) }\SpecialCharTok{==} \DecValTok{5}\NormalTok{)\{}
\NormalTok{      sum\_5\_given\_4[i] }\OtherTok{=} \DecValTok{1}
\NormalTok{    \}}
\NormalTok{\}}

\CommentTok{\# determine the proportion of rolls that contain a 4 that summed to 5}
\NormalTok{result }\OtherTok{\textless{}{-}} \FunctionTok{sum}\NormalTok{(sum\_5\_given\_4) }\SpecialCharTok{/} \FunctionTok{sum}\NormalTok{(contains\_4)}
\end{Highlighting}
\end{Shaded}

\begin{Shaded}
\begin{Highlighting}[]
\NormalTok{expected\_value }\OtherTok{\textless{}{-}} \DecValTok{2} \SpecialCharTok{/} \DecValTok{11}
\NormalTok{result\_error }\OtherTok{\textless{}{-}} \FunctionTok{abs}\NormalTok{(}\DecValTok{1} \SpecialCharTok{{-}}\NormalTok{ (result}\SpecialCharTok{/}\NormalTok{expected\_value)) }\SpecialCharTok{*} \DecValTok{100}

\FunctionTok{print}\NormalTok{(}\FunctionTok{sprintf}\NormalTok{(}\StringTok{"The average of \%d simulations is \%f which is within \%.3f\%\% of the expected result of 2/11"}\NormalTok{, sims, result, result\_error))}
\end{Highlighting}
\end{Shaded}

\begin{verbatim}
[1] "The average of 1000000 simulations is 0.183224 which is within 0.773% of the expected result of 2/11"
\end{verbatim}



\end{document}
